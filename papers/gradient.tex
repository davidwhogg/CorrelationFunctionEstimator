% This file is part of the CosmologicalGradient project.
% Copyright 2018 the authors.

\documentclass[12pt, letterpaper]{article}

% margins
\addtolength{\topmargin}{-0.75in}
\addtolength{\textheight}{1.50in}

\begin{document}

\paragraph{abstract:}
% Context:
Cosmological homogeneity is well established in galaxy surveys in a mean-density
sense.
At the same time, there are suggestions of small anisotropies in clustering power in
the microwave background data, and inflation-like theories might naturally produce
gradients in physical parameters on extremely large scales.
% Aims:
Here we ask about the statistical homogeneity of the clustering properties of galaxies.
That is: Is there a large-scale gradient in the correlation function of galaxies
apparent in the well-observed part of the Hubble volume?
% Method:
We generate an estimator (implicitly part of a large family of estimators)
that delivers simulateously the galaxy--galaxy auto-correlation function
(or indeed any cross-corelation function)
and it's spatial gradient.
The estimator is general; we apply it to the XXX sample of the YYY survey.
% Results:
We don't find any significant gradient at the ZZZ level.
The implications of the result are discussed in the context of contemporary ideas
about physical cosmology and inflation.

\section{Introduction}

What do we know about galaxy homogeneity?

What are the suggestions in the CMB or other places for anisotropies?

What are the theoretical ideas about super-horizon gradients in physical laws?

\section{Methods}

Recall:
\begin{eqnarray}
\xi_k &=& \frac{DD_k - 2\,DR_k + RR_k}{RR_k}
\\
DD_k &\equiv& \sum_{n n'} i(a_k < |x_n - x_{n'}| < b_k)
\\
DR_k &\equiv& \sum_{n m} i(a_k < |x_n - x_m| < b_k)
\\
RR_k &\equiv& \sum_{m m'} i(a_k < |x_m - x_{m'}| < b_k)
\quad ,
\end{eqnarray}
where...

\section{Data and results}

\section{Discussion}

\end{document}
